\documentclass[a4paper, 12pt]{article}

% Margins
\topmargin=-0.45in
\evensidemargin=0in
\oddsidemargin=0in
\textwidth=6.5in
\textheight=9.0in
\headsep=0.25in 

%package usage
\PassOptionsToPackage{svgnames}{xcolor}
\usepackage[colorlinks=true, citecolor=blue, linkcolor=purple]{hyperref}
\usepackage[english]{babel}
\usepackage[latin1]{inputenc}
\usepackage{enumitem}
\usepackage{indentfirst}
\usepackage{colortbl}
\usepackage{longtable}
\usepackage{animate}
\usepackage{threeparttablex}
\usepackage{etoolbox}
\usepackage{rotating}
\usepackage{multirow}
\usepackage{pdflscape}
\usepackage{tablefootnote}
\usepackage[table,xcdraw]{xcolor}
\usepackage{amsmath}
\usepackage{flafter} 
\usepackage{dcolumn} 
\usepackage{natbib}
\usepackage{rotating}	
\usepackage{adjustbox}
\usepackage{amsthm}
\usepackage{graphicx}
\usepackage{amssymb}
\usepackage{tcolorbox}
\usepackage{lipsum}
\usepackage{tikz}
\usepackage{tabularx}
\tcbuselibrary{skins,breakable}
\usetikzlibrary{shadings,shadows}
\usepackage{threeparttable}
\usepackage{subfig}
\usepackage{setspace}
\usepackage{booktabs}
\usepackage{placeins}
\usepackage{enumitem}
\usepackage{natbib}
\usepackage{filecontents}
\usepackage[encoding,filenameencoding=utf8,extendedchars,space]{grffile}
\newcommand{\addfig}[2]{\begin{center}
			\includegraphics[width=#1\textwidth]{#2}
	\end{center}
}

%\usepackage[capposition=top]{floatrow}
%\usepackage[colorinlistoftodos]{todonotes}
\newcommand{\expect}[2]{\mathbb{E}_{#2}\left(#1\right)}

\newtheorem{theorem}{Theorem}[section]
\newtheorem{corollary}{Corollary}[theorem]
\newtheorem{proposition}[theorem]{Proposition}

\newcommand{\alert}[1]{{\textbf{\color{red}#1}}}
%general commands

\newcommand{\bcite}{\begin{quote}}
\newcommand{\ecite}{\end{quote}}
\newcommand{\beqns}{\begin{eqnarray*}}
\newcommand{\eeqns}{\end{eqnarray*}}
\newcommand{\beqn}{\begin{eqnarray}}
\newcommand{\eeqn}{\end{eqnarray}}
\newcommand{\benu}{\begin{enumerate}}
\newcommand{\eenu}{\end{enumerate}}
\newcommand{\bitem}{\begin{itemize}}
\newcommand{\eitem}{\end{itemize}}
\newcommand{\smallGap}{\vspace{.25cm}}

\newenvironment{block}[1]{%
	\tcolorbox[beamer,%
	noparskip,breakable,
	colback=LightGreen,colframe=DarkGreen,%
	colbacklower=LimeGreen!75!LightGreen,%
	title=#1]}%
{\endtcolorbox}


\newcommand{\sym}[1]{\rlap{#1}}% Thanks David Carlisle

\usepackage{siunitx}
\sisetup{
	detect-mode,
	group-digits		= false,
	input-symbols		= ( ) [ ] - +,
	table-align-text-post	= false,
	input-signs             = ,
}

%mathematical commands
\newcommand{\problemIndustries}{(A_t\cup \Omega_t)}
\newcommand{\red}[1]{{\color{red}#1}}
\newcommand{\normalIndustries}{\problemIndustries^C}
\newcommand{\sumNormalIndustries}{\sum_{j\in\normalIndustries}}
\newcommand{\sumProblemIndustries}{\sum_{j\in\problemIndustries}}
\newcommand{\sumWithinNormal}{\sum_{j\in\normalIndustries\cap \Gamma_t}}
\newcommand{\sumWithinProblem}{\sum_{j\in\normalIndustries\cap \Gamma_t^C}}
\newcommand{\ser}[1]{s_{er#1}}
\newcommand{\cov}{\text{cov}}
\newcommand{\explain}[2]{\underbrace{#1}_{\parbox{\widthof{\ensuremath{#1}}}{\footnotesize\raggedright #2}}}
\newcommand{\lfpthresh}[1]{\underline{\chi_{#1 lt}}}
\newcommand{\crho}{\frac{\sigma-1}{\sigma}}
\newcommand{\crhoinv}{\frac{\sigma}{\sigma-1}}

\newcommand{\lquote}[3][]{\bcite #2 \citep[#1]{#3} \ecite}

\newcommand{\laquote}[3][]{\bcite #2 \citepalias[#1]{#3} \ecite}
\usepackage{tabularx}
\defcitealias{NationalAcademyofSciences.CommitteeonOccupationalClassificationandAnalysis.1971}{National Academy of Sciences, 1971}


\title{Code documentation}
\author{C\'esar Garro-Mar\'in\thanks{Boston University, email: \href{mailto:cesarlgm@bu.edu}{cesarlgm@bu.edu}}} 
\begin{document}
\maketitle

\newcommand{\ntimes}{4 }

\begin{figure}
	\caption{Change in employment shares: 2001-2017}
	\addfig{1}{../results/figures/direction_triangle}
\end{figure}

\FloatBarrier
\begin{figure}
\caption{Abstract, Routine and Manual}
\subfloat[Low-mid]{\animategraphics[loop,controls,width=\linewidth]{1}{../results/figures/border_triangle_mra12_}{2001}{2017}}
\end{figure}
\begin{figure}
\ContinuedFloat
\subfloat[Low-High]{\animategraphics[loop,controls,width=\linewidth]{1}{../results/figures/border_triangle_mra13_}{2001}{2017}}
\end{figure}
\begin{figure}
\ContinuedFloat
\subfloat[Mid-High]{\animategraphics[loop,controls,width=\linewidth]{1}{../results/figures/border_triangle_mra23_}{2001}{2017}}
\end{figure}

\begin{figure}
	\caption{Social, Routine and Manual}
	\subfloat[Low-mid]{\animategraphics[loop,controls,width=\linewidth]{1}{../results/figures/border_triangle_mrs12_}{2001}{2017}}
\end{figure}
\begin{figure}
	\ContinuedFloat
	\subfloat[Low-High]{\animategraphics[loop,controls,width=\linewidth]{1}{../results/figures/border_triangle_mrs13_}{2001}{2017}}
\end{figure}
\begin{figure}
	\ContinuedFloat
	\subfloat[Mid-High]{\animategraphics[loop,controls,width=\linewidth]{1}{../results/figures/border_triangle_mrs23_}{2001}{2017}}
\end{figure}


\begin{figure}
	\caption{Abstract, Routine and Social}
	\subfloat[Low-mid]{\animategraphics[loop,controls,width=\linewidth]{1}{../results/figures/border_triangle_sra12_}{2001}{2017}}
\end{figure}
\begin{figure}
	\ContinuedFloat
	\subfloat[Low-High]{\animategraphics[loop,controls,width=\linewidth]{1}{../results/figures/border_triangle_sra13_}{2001}{2017}}
\end{figure}
\begin{figure}
	\ContinuedFloat
	\subfloat[Mid-High]{\animategraphics[loop,controls,width=\linewidth]{1}{../results/figures/border_triangle_sra23_}{2001}{2017}}
\end{figure}


\begin{figure}
	\caption{Abstract, Social and Manual}
	\subfloat[Low-mid]{\animategraphics[loop,controls,width=\linewidth]{1}{../results/figures/border_triangle_msa12_}{2001}{2017}}
\end{figure}
\begin{figure}
	\ContinuedFloat
	\subfloat[Low-High]{\animategraphics[loop,controls,width=\linewidth]{1}{../results/figures/border_triangle_msa13_}{2001}{2017}}
\end{figure}
\begin{figure}
	\ContinuedFloat
	\subfloat[Mid-High]{\animategraphics[loop,controls,width=\linewidth]{1}{../results/figures/border_triangle_msa23_}{2001}{2017}}
\end{figure}

\FloatBarrier
%
%\section{Estimating $\theta$}
%\label{sec:model}
%%Fix the notation here
%We use two key equations from the model:
%\beqn
%\sum_{i=1}^I\theta_i^eS_{i}^e(J)&=&1 \\
%d\ln f^e(J)&=&\frac{\varepsilon}{1-\varepsilon}\sum_i\theta_i^e(d\ln A_i - K^e)S_i^e(J) \label{eq:empshare}
%\eeqn
%where $e\in\{1,2,3\}$ indexes the education group and $i$ indexes the skill. We impose several normalizations to the model:
%\bitem 
%\item Skill acquisition costs are 1 for the low education group $\theta_i^1=1$.
%\item Manual acquisition costs are 1 for all education groups $\theta_1^e=1$.
%\eitem
%
%
%We do not observe the skill indexes $S_{i}^e(J)$. We estimate them using information from 22 questions from the SES survey. These questions typically ask respondents to rate how important a given task or job trait is for their job. These ratings go from 1 to 5, 5 being very important. We combine all these answers into an index as follows:
%\beqn
%S_{i}(J)=\sum_{j=1}^{||i||}\alpha_{ij}\sum_{l=1}^5c_{ijl}1_{d_{ij}=l}
%\eeqn
%where $d_{jm}\in\{1,2,3,4,5\}$ is the individual's answer to the SES question $ij$. I abused notation  by indicating the number of SES questions in index $i$ with $||i||$.  $\alpha_{ij}$ and $c_{ijl}$ are parameters to estimate. The scales $c_{ijl}$  are the numeric values we give to each answer in the Likert scale. The $\alpha_{ij}$ are weights we give to each question within the index. For all questions we normalize the lowest scale value to zero ($c_{ij1}=0, \forall i,j$) and highest value to 1 ($c_{ij5}=0,\forall i,j$). The only restriction on the weights  $\alpha_{ij}$ is that they must be non-negative.
%
%\subsection{Data I use for estimation}
%The data I feed to the algorithm satisfies two restrictions:
%\benu
%	\item I restrict to jobs that I observe as \red{\textbf{core}} in any two consecutive SES years.
%	\item As of now, I \red{\textbf{am not}} applying any restriction by education group. 
%	
%	So, for example I am including observations of low education individuals that are in high-education core jobs.
%\eenu
%
%\subsection{Procedure}
%
%We start by guessing values for the weights $\alpha_{jm}$ and the Likert scales $c_{jml}$. This gives us an initial guess for the skill indexes $S_{\theta,m}(J)$.
%
%\paragraph{Step 1: estimate skill acquisition costs:} given the guess for $S_{\theta,m}(J)$ we estimate the skill acquisition costs $\theta_{i}^e$ using the empirical analogous of equation \eqref{eq:empshare}.
%\beqns
%d\ln f^e(J)&=&\sum_i\beta_i^eS_i^e(J) + \lambda_I+v^e(J)
%\eeqns
%In this equation we constrain the $\theta_{i}$ to be positive by estimating the following non-linear equation:
%\beqns
%d\ln f^1(J)&=&\sum_i\beta_i^1S_i^e(J) + v^1(J)\\
%d\ln f^e(J)&=&\beta_1^eS_1^e(J)+\sum_{i=2}^4(\sqrt{\theta_i^e})^2\left(\beta_{i}^1-\beta_{1}^1+\beta_{1}^e\right)S_i^e(J) + v^e(J), e>1\\
%\eeqns
%%We can compute the skill acquisition costs out of the $\beta_i^e$ using:
%\beqn
%\theta_i^e=\frac{\beta_i^e}{\beta_i^1-\beta_1^1+\beta_1^e} \label{eq:empirical}
%\eeqn
%
%\paragraph{Step 2: estimate Likert scales and question weights} we use the $\theta_i^e$ estimated in the step above to compute the Likert scales and the question weights. We choose scales and weights to minimize the MSE from equation \eqref{eq:skill_sum}: 
%
%\beqns
%\min_{\alpha_{jm},c_{jml}}\frac{1}{N}\left[\sum_{m=1}^I\theta_j^eS_{m}^e-1\right]^2 \text{ s.t. }  S_{m}^e=\sum_{j=1}^{||m||}\alpha_{jm}\sum_{l=1}^5c_{jml}1_{d_{ijm}=l}
%\eeqns
%
%\paragraph{Iterate until $\theta$ converges}: we iterate this procedure until the skill acquisition costs converge: 
%\beqns
%||\Theta_n-\Theta_{n-1}||<0.01
%\eeqns
%
%where $\Theta_n$ is the vector of skill acquisition costs estimated in step $n$\footnote{You could argue that 0.01 is a loose tolerance threshold is large, but I was getting bored of having the code running. I just wanted to see a likely convergence point.}.
%
%\section{$\theta$s from regressions out of simple average skill indexes}
%Here I show estimated skill acquisition costs when I compute them using simple average indexes. These $\theta_i$ come from the regression:
%\beqn
%d\ln f^e(J)&=&\sum_i\beta_i^eS_i^e(J)\\
%\eeqn
%where:
%\beqns
%\beta_{i}^e=\frac{\varepsilon}{1-\varepsilon}\theta_i^e(d\ln A_i - K^e)
%\eeqns
%I define the indexes as follows:
%\beqns
%S_i(J)&=&\frac{\tilde{S}_i(J)}{\sum_k^K\tilde{S}_k(J) } \\
%\eeqns
%where $S_k(J)$ is the simple average of the scores I assigned to each SES question:
%\beqns
%S_{i}(J)=\frac{1}{||i||}\sum_{j=1}^{||i||}\sum_{l=1}^5c_{ijl}1_{d_{ij}=l}\\
%\eeqns
%remember that the SES questions have possible answer going from 1 to 5. I normalized these answers $c_{ijl}$  to be between zero and one.
%\beqns
%c_{ijl}=\frac{l-1}{4}
%\eeqns
%\subsection{Results}
%Weighted result weights observation by the occupation-years cells size. The implied thetas are in these files:
%\bitem
%\item Unweighted $\theta$s: \href{https://www.dropbox.com/s/epmal84fzmnwiam/unweighted_thetas.txt?dl=0}{click here}
%\item Weighted $\theta$s: \href{https://www.dropbox.com/s/bkq8o6zcjmgpjf8/weighted_thetas.txt?dl=0}{click here}
%\eitem
%\FloatBarrier
%\begin{center}
\begin{threeparttable}[!h]
\caption{Estimates of $\beta_{i}^e$}
\begin{tabular}{lcc}
\toprule
\toprule
&\multicolumn{1}{c}{\textbf{Unweighted}}&\multicolumn{1}{c}{\textbf{Weighted}} \\
\textbf{}&\multicolumn{1}{c}{(1)}&\multicolumn{1}{c}{(2)} \\
\midrule
manual1             &        0.03&        0.03\\
                    &      (0.11)&      (0.11)\\
abstract1           &       -0.69&       -0.69\\
                    &      (0.26)&      (0.26)\\
social1             &       -0.01&       -0.01\\
                    &      (0.29)&      (0.29)\\
routine1            &       -0.19&       -0.19\\
                    &      (0.18)&      (0.18)\\
manual2             &        0.58&        0.58\\
                    &      (0.28)&      (0.28)\\
abstract2           &       -2.78&       -2.78\\
                    &      (0.60)&      (0.60)\\
social2             &        2.31&        2.31\\
                    &      (0.66)&      (0.66)\\
routine2            &        0.56&        0.56\\
                    &      (0.37)&      (0.37)\\
manual3             &        0.06&        0.06\\
                    &      (0.13)&      (0.13)\\
abstract3           &        0.17&        0.17\\
                    &      (0.15)&      (0.15)\\
social3             &       -0.11&       -0.11\\
                    &      (0.18)&      (0.18)\\
routine3            &       -0.62&       -0.62\\
                    &      (0.24)&      (0.24)\\
n\_occupations       &            &            \\
N                   &       4,687&       4,687\\
r2                  &        0.07&        0.07\\
\bottomrule
\bottomrule
\end{tabular}
\begin{tablenotes}
\item \footnotesize \textit{Notes:} Standard errors clustered at the occupation level. Estimates include industry by education fixed-effects. Table generated on  8 Mar 2022 at 16:33:20.
\end{tablenotes}
\end{threeparttable}
\end{center}

%\begin{table}[htbp]\centering \caption{$\theta_{ie}$ estimates} \begin{tabular}{l*{4}{c}} \toprule
            &      manual&     routine&      social&    abstract\\
\midrule
Low         &        0.46&        0.40&        0.37&        0.64\\
Mid         &        0.65&        0.29&        0.20&        0.67\\
High        &        0.59&        0.45&        0.25&        0.61\\
\bottomrule
\end{tabular}
\end{table}

%\FloatBarrier
%%
%%\section{Simulating data}
%%I simulate data based on equations:
%%\beqn
%%\sum_{i=1}^I\theta_i^eS_{i}^e(J)&=&1 \label{eq:skill_sum}\\
%%d\ln f^e(J)&=&\frac{\varepsilon}{1-\varepsilon}\sum_i\theta_i^e(d\ln A_i - K^e)S_i^e(J) \label{eq:empshares}
%%\eeqn
%%\subsection{Procedure}
%%\subsubsection{Assumptions}
%%I assume a world of 2 education groups, 2 skills, and each skill is made up of a unique dummy variable:
%%\newline
%%
%%\noindent\textbf{Model parameters:}
%%\bitem 
%%	\item $\varepsilon=0.5$
%%	\item $d\ln A=\begin{pmatrix}
%%		3 & -2 \\
%%		3 & -2
%%	\end{pmatrix}$
%%	\item $K^e=2$
%%	\item $\theta=\begin{pmatrix}
%%		1 & 1 \\
%%		1 & 0.5
%%	\end{pmatrix}$
%%\eitem 
%%In this world equations equation \eqref{eq:skill_sum} holds at the job level\footnote{It makes a difference whether I assume this is true at the job level, or at the individual level!} with some noise so that:
%%\beqns
%%	S_1^e(J)=1-\theta_2^eS_2^e(J)+\nu(J) \label{eq:sim_prob}
%%\eeqns
%%because the skills are dummy variables, then $S_i^e(J)$ is simply the probability that the skill variable is equal to 1 for job $J$ and education $e$. Given $S_2^e(J)$ and $\nu(J)$, equation \eqref{eq:sim_prob} defines the probabilities to generate data for skill 1. Throughout I assume $p_2^1(J)=\frac{1}{8}$, $p_2^2(J)=\frac{7}{8}$ and $v(J) \sim U[  -m,m ]$. I chose $m$ so that the probabilities are well defined.
%%
%%Next, I generate employment share data using: 
%%\beqns
%%d\ln f^e(J)&=&\frac{\varepsilon}{1-\varepsilon}\sum_i\theta_i^e(d\ln A_i - K^e)S_i^e(J)+\kappa(J)
%%\eeqns
%%where $\kappa(J)\sim N(0,1)$.
%%
%%\paragraph{Estimated $\theta$}
%%\bitem 
%%\item $\theta=\begin{pmatrix}
%%	1 & 1 \\
%%	1 & 0.53
%%\end{pmatrix}$
%%\eitem 
%
%
%%
%%\red{Put this in a proof later}
%%The normalization on the manual index implies:
%%\beqns
%%\beta_i^1=d\ln A_i-K^e
%%\eeqns
%%therefore:
%%\beqns
%%d\ln A_1&=&\beta_1^1+K\\
%%K^e&=&-\beta_1^e+\beta_1^1+K^e
%%\eeqns
%%Now, from the low-education normalizations we have:
%%\beqns
%%d\ln A_i&=&\beta_i^1+K^e\\	
%%\eeqns
%%Therefore:
%%\beqns
%%\theta_i^e=\frac{\beta_i^e}{\beta_i^1-\beta_1^1+\beta_1^e}
%%\eeqns
%%
%%\section{Defining education groups}
%%	Our current results group education levels into three broad groups that I will often call Low, Mid, and High.
%%% Table generated by Excel2LaTeX from sheet 'Sheet1'
\begin{table}[htbp]
	\centering
	\caption{Add caption}
	\begin{tabular}{ll}
		\toprule
		\textbf{ Label } & \textbf{ GCSE qualification level } \\
		\midrule
		Low   & Below GCSE A \\
		Mid   & GCSE A* / trade qualification \\
		High  & Bachelor +  \\
		\bottomrule
		\bottomrule
	\end{tabular}%
	\label{tab:addlabel}%
\end{table}%

%%	
%%
%%\section{Classifying jobs}
%%We say that an occupation $j$ is a core job of education group $e$ if two conditions are met:
%%\benu	 
%%	\item Education group $e$ is overrepresented in the occupation relative to the overall population. That is:
%%	\beqns
%%		s_e(j)\geq\overline{s}_e
%%	\eeqns
%%	where $s_e(j)$ denotes the employment share of the education group $e$ in job $j$, and $\overline{s}_e$ is its employment share in the population.
%%	\item The employment share of group $e$ in job $j$ is at least \ntimes the employment share of any other education group that is overrepresented in the occupation.
%%	\beqns
%%		s_e(j)\geq\ntimes s_{e'}(j)
%%	\eeqns
%%	for any other education group $e'$ such that $	s_{e'}(j)\geq\overline{s}_{e'}$.
%%\eenu
%%
%%%\section{Computing $\theta$s}
%%%\subsection{Data I use}
%%%First I restrict data to only:
%%%\benu 
%%%\item occupations that are core jobs in two consecutive SES-waves.
%%%\item people with education levels matching the job-classification. For example, I restrict to observations of individuals with low-education in low-education core-jobs.
%%%\eenu 
%%%Using this restricted dataset, I occupational employment shares by education level:
%%%\beqns
%%%	s_e(j)=\frac{l_e(j)}{\sum_{j'}l_e(j')}
%%%\eeqns
%%%where $l$ denotes employment and the summation is over jobs that stay in the core of education group $e$ in two consecutive SES-waves.
%%%
%%%
%%%\section{Solution procedure}
%%%
%%%Out of equation 32 we have:
%%%\beqns
%%%	\frac{\partial \ln f_\theta(J)}{\partial A_i}-\frac{\partial \ln f_\theta(J')}{\partial A_i}&=&\frac{\varepsilon}{\varepsilon-1}\left[\frac{\ln y^\star_\theta(J)}{\partial \ln A_i}-\frac{\ln y^\star_\theta(J')}{\partial \ln A_i}\right]
%%%\eeqns
%%%Moreover, out of question 44 we have
%%%\beqns
%%%\frac{\partial \ln y^\star_\theta(J)}{\partial \ln A_i}&=&\theta_iS^\star_{\theta,i}(J)
%%%\eeqns
%%%Plugging into 32 we have:
%%%\beqn
%%%\label{eq:plug}
%%%\frac{\partial \ln f_\theta(J)}{\partial A_i}-\frac{\partial \ln f_\theta(J')}{\partial A_i}&=&\frac{\varepsilon}{\varepsilon-1}\left[\theta_iS^\star_{\theta,i}(J)-\theta_iS^\star_{\theta,i}(J')\right]
%%%\eeqn
%%%Thus:
%%%\beqns
%%%d\ln f_\theta(J)-d\ln f_\theta(J')=\frac{\varepsilon}{\varepsilon-1}\sum_i\left[\theta_iS^\star_{\theta,i}(J)-\theta_iS^\star_{\theta,i}(J')\right]d\ln A_i+\theta_MS^\star_{\theta,M}(J)-\theta_MS^\star_{\theta,M}(J')
%%%\eeqns
%%%Summing over jobs and dividing by the number of jobs we have:
%%%\beqns
%%%d\ln f_\theta(J)-\overline{d\ln f_\theta(J)}=\frac{\varepsilon}{\varepsilon-1}\sum_i\left[\theta_iS^\star_{\theta,i}(J)-\theta_i\overline{S^\star_{\theta,i}(J)}\right]d\ln A_i
%%%\eeqns
%%%This equation calls for the following regression specification:
%%%\beqns
%%%	d\ln f_\theta(J)-\overline{d\ln f_\theta(J)}&=&\alpha_\theta+\sum_{i}\beta_{\theta,i}(S^\star_{\theta,i}(J)-\theta_i\overline{S^\star_{\theta,i}(J)})+\nu_{\theta}(J)
%%%\eeqns
%%%Then:
%%%\bitem
%%%	\item Under the assumption that $\theta_i=1$, $\frac{\varepsilon}{\varepsilon-1}d\ln A_i$ is identified out of the low education group.
%%%	\item Rest of education groups identify $\theta_i$. 
%%%\eitem 
%%%
%%%
%%%
%%%
%%%
%%%\subsection{Procedure}
%%%\benu
%%%	\item Guess $S_{\theta,i}(J)$.
%%%	\item Estimate $\theta_i$ out of core jobs. %\red{What is the reference job? The average for that education group?}
%%%	\item Given $\theta_i$ estimate $S_{\theta,i}(J)$.
%%%	\item Return to 1.
%%%\eenu
%%
%%\subsection{What functions do I need to write}
%%\subsubsection{Estimation of $\theta_i$}
%%Let $y_\theta$ be the $J\times 1$ vector containing the vector of $d\ln f_\theta(J)-d\ln f_\theta(J')$. Let $S_\theta$ the $J\times I$ matrix of skill indexes $S^\star_{\theta,i}(J)-S^\star_{\theta,i}(J')$. Then:
%%\beqns
%%	\beta_\theta=\frac{\epsilon}{\epsilon-1}[\theta_1d\ln A_1 \dots \theta_Id\ln A_I]'
%%\eeqns
%%I estimate $\beta_\theta$ by OLS:
%%\beqns
%%	\beta_\theta=(X_\theta'X_\theta)^{-1}X_\theta y_\theta
%%\eeqns
%%Using the appropriate block diagonal matrix I can estimate all the vectors at the same time. For this I need:
%%\bitem
%%\item The usual OLS function
%%\item The function to block diagonalize the matrix that I already wrote.
%%\eitem 
%%Next, I need to back out the $\theta_i$. For this I need to do:
%%\beqns
%%	\beta_1=\frac{\epsilon}{\epsilon-1}[d\ln A_1 \dots d\ln A_I]'
%%\eeqns
%%Then:
%%\beqns
%%	\theta=\beta_\theta\oslash\beta_1
%%\eeqns
%%For this I need:
%%\bitem
%%	\item Function splitting the vector by education level.
%%	\item Function estimating $\beta_{\theta}$: {\tt estimate\_beta\_theta}
%%	\item Function estimating the $\theta$ {\tt estimate\_theta}.
%%	\item Function estimating averages of skill indexes by education level: {\tt average\_skill\_use}.
%%\eitem 
%%\subsubsection{How do I estimate the scales then?}
%%There are a set of $O$ skill questions in the SES survey that we have partitioned into $M$ mutually exclusive groups that we index by $m$. Within each partition, we index the skill questions by $j$. Let $d_{ijm}$ be individual's $i$ answer for the skill question $jm$. $d_{ijm}\in\{1,2,3,4,5\}$. The problem is then:
%%\beqns
%%	\min_{\alpha_{jm},c_{jml}}\frac{1}{N}\left[\sum_{m=1}^I\theta_jS_{\theta,m}\right]^2 \text{ s.t. }  S_{\theta,m}=v
%%\eeqns
%%\bitem
%%	\item I think this is mostly done. I just have to modify the loss function for this.
%%\eitem 
\bibliographystyle{apalike}
%\bibliography{../../../../CentralLibrary/Papers/library}{}

\end{document}

