\documentclass[a4paper, 12pt]{article}

% Margins
\topmargin=-0.45in
\evensidemargin=0in
\oddsidemargin=0in
\textwidth=6.5in
\textheight=9.0in
\headsep=0.25in 

%package usage
\PassOptionsToPackage{svgnames}{xcolor}
\usepackage[colorlinks=true, citecolor=blue, linkcolor=purple]{hyperref}
\usepackage[english]{babel}
\usepackage[latin1]{inputenc}
\usepackage{enumitem}
\usepackage{indentfirst}
\usepackage{colortbl}
\usepackage{longtable}
\usepackage{animate}
\usepackage{threeparttablex}
\usepackage{etoolbox}
\usepackage{rotating}
\usepackage{multirow}
\usepackage{pdflscape}
\usepackage{tablefootnote}
\usepackage[table,xcdraw]{xcolor}
\usepackage{amsmath}
\usepackage{flafter} 
\usepackage{dcolumn} 
\usepackage{natbib}
\usepackage{rotating}	
\usepackage{adjustbox}
\usepackage{amsthm}
\usepackage{graphicx}
\usepackage{amssymb}
\usepackage{tcolorbox}
\usepackage{lipsum}
\usepackage{tikz}
\usepackage{tabularx}
\tcbuselibrary{skins,breakable}
\usetikzlibrary{shadings,shadows}
\usepackage{threeparttable}
\usepackage{subfig}
\usepackage{setspace}
\usepackage{booktabs}
\usepackage{placeins}
\usepackage{enumitem}
\usepackage{natbib}
\usepackage{filecontents}
\usepackage[encoding,filenameencoding=utf8,extendedchars,space]{grffile}
\newcommand{\addfig}[2]{\begin{center}
			\includegraphics[width=#1\textwidth]{#2}
	\end{center}
}

%\usepackage[capposition=top]{floatrow}
%\usepackage[colorinlistoftodos]{todonotes}
\newcommand{\expect}[2]{\mathbb{E}_{#2}\left(#1\right)}

\newtheorem{theorem}{Theorem}[section]
\newtheorem{corollary}{Corollary}[theorem]
\newtheorem{proposition}[theorem]{Proposition}

\newcommand{\alert}[1]{{\textbf{\color{red}#1}}}
%general commands

\newcommand{\bcite}{\begin{quote}}
\newcommand{\ecite}{\end{quote}}
\newcommand{\beqns}{\begin{eqnarray*}}
\newcommand{\eeqns}{\end{eqnarray*}}
\newcommand{\beqn}{\begin{eqnarray}}
\newcommand{\eeqn}{\end{eqnarray}}
\newcommand{\benu}{\begin{enumerate}}
\newcommand{\eenu}{\end{enumerate}}
\newcommand{\bitem}{\begin{itemize}}
\newcommand{\eitem}{\end{itemize}}
\newcommand{\smallGap}{\vspace{.25cm}}

\newenvironment{block}[1]{%
	\tcolorbox[beamer,%
	noparskip,breakable,
	colback=LightGreen,colframe=DarkGreen,%
	colbacklower=LimeGreen!75!LightGreen,%
	title=#1]}%
{\endtcolorbox}


\newcommand{\sym}[1]{\rlap{#1}}% Thanks David Carlisle

\usepackage{siunitx}
\sisetup{
	detect-mode,
	group-digits		= false,
	input-symbols		= ( ) [ ] - +,
	table-align-text-post	= false,
	input-signs             = ,
}

%mathematical commands
\newcommand{\problemIndustries}{(A_t\cup \Omega_t)}
\newcommand{\red}[1]{{\color{red}#1}}
\newcommand{\normalIndustries}{\problemIndustries^C}
\newcommand{\sumNormalIndustries}{\sum_{j\in\normalIndustries}}
\newcommand{\sumProblemIndustries}{\sum_{j\in\problemIndustries}}
\newcommand{\sumWithinNormal}{\sum_{j\in\normalIndustries\cap \Gamma_t}}
\newcommand{\sumWithinProblem}{\sum_{j\in\normalIndustries\cap \Gamma_t^C}}
\newcommand{\ser}[1]{s_{er#1}}
\newcommand{\cov}{\text{cov}}
\newcommand{\explain}[2]{\underbrace{#1}_{\parbox{\widthof{\ensuremath{#1}}}{\footnotesize\raggedright #2}}}
\newcommand{\lfpthresh}[1]{\underline{\chi_{#1 lt}}}
\newcommand{\crho}{\frac{\sigma-1}{\sigma}}
\newcommand{\crhoinv}{\frac{\sigma}{\sigma-1}}

\newcommand{\lquote}[3][]{\bcite #2 \citep[#1]{#3} \ecite}

\newcommand{\laquote}[3][]{\bcite #2 \citepalias[#1]{#3} \ecite}
\usepackage{tabularx}
\defcitealias{NationalAcademyofSciences.CommitteeonOccupationalClassificationandAnalysis.1971}{National Academy of Sciences, 1971}


\title{Code documentation}
\author{C\'esar Garro-Mar\'in\thanks{Boston University, email: \href{mailto:cesarlgm@bu.edu}{cesarlgm@bu.edu}}} 

\begin{document}
\maketitle

\section{GMM standard errors}
I minimize the quadratic form:
\beqn
Q=\left(\frac{1}{N}\sum_{i=1}^{N}\psi(w_i,c)\right)'A\left(\frac{1}{N}\sum_{i=1}^{N}\psi(w_i,c)\right)
\eeqn 
The GMM estimates have a distribution of:
\beqns
\sqrt{N}(\hat\mu-\mu)\rightarrow N(0,\tilde{V})
\eeqns
where $\bar{V}=(D'AD)^{-1}D'AVAD(D'AD)^{-1}$. Here:
\bitem
	\item $D$ is the model gradient.
	\item $V$ is defined as:
	\beqns
		V=\lim_{N\rightarrow\infty}\frac{1}{{N}}\sum_{i=1}^N\expect{\psi(w_i,\mu)\psi(w_i,\mu)'}{}
	\eeqns
	\item $A$ is the weighting matrix $(Z'Z)^{-1}$
\eitem

\subsection{Estimating the variance matrix}
I will now describe how do I estimate each component of the variance covariance matrix $A$:
\bitem 
	\item \textbf{Estimating $V$:} our estimate is:
	\beqns
	\hat{V}=\frac{1}{{N}}\sum_{i=1}^N\expect{\psi(w_i,\hat{\mu})\psi(w_i,\hat{\mu})'}{}
	\eeqns
	we compute this estimate this component in the function {\tt estimate\_v}
	\item \textbf{Estimating $W$:} our weighting matrix is simple $(Z'Z)^{-1}$.
	\item \textbf{Estimating $D$:} this is the gradient of the model's errors. We compute this matrix in the function {\tt estimate\_d}.
\eitem


\appendix
\section{Notation}
Our GMM model is based on the moment equations:
\beqns
	\expect{\psi(w_i,\mu)}{}=0	
\eeqns
where:
\bitem
\item $w_i$ is the vector of data for observation $i$.
\item $\psi$ is a $P\times 1$ vector of functions.
\item $\mu$ is an $R\times 1$ vector of parameters. 
\eitem 
Our estimate is the solution to the problem:
\beqns
	\hat{\mu}=\arg\min_c\left(\frac{1}{N}\sum_{i=1}^{N}\psi(w_i,c)\right)'A\left(\frac{1}{N}\sum_{i=1}^{N}\psi(w_i,c)\right)
\eeqns
which we can write as:
\beqns
	\hat{\mu}=\arg\min_c\left(\frac{1}{N}\varepsilon(c)'Z\right)A\left(\frac{1}{N}Z'\varepsilon(c)\right)
\eeqns
where $\varepsilon(c)$ is the $N\times 1$ vector of errors of the model, and $Z$ is the $N\times P$ matrix of instruments. Here $N=N_1+N_2+N_3$ is the total number of observations and $N_i$ denotes the number of observations that belong to equation $i$. 
\bibliographystyle{apalike}
%\bibliography{../../../../CentralLibrary/Papers/library}{}

\subsection{The gradient:}
Our estimate of the gradient $D$ is:
\beqns
	\hat{D}=\frac{1}{N}\sum_{i=1}^N\frac{\partial \psi(w_l,\hat{\mu})}{\partial c'}
\eeqns
now, let us describe the form of this gradient in more detail. Each element of $\psi(w_l,c)$ is the function $z_{lp}\varepsilon_i(c)$, where $\varepsilon_l(c)$ is the error term for the $l$-th observation.   

To avoid making the notation a mess, define the functions $J(l)$, $E(l)$, $T(l)$, and $I(l)$ which return the job, education level, year, and skill that correspond to observation $l$. In addition, for the third equation observations I define as $EET(l)$ as the function that returns the education pair-year cell of the observation.
\beqns
	\varepsilon_l(\mu)=\left\{\begin{array}{cc}
		\Delta \overline{ \ln S_{ijet+1}}-\sum_k \theta_{ke}\overline{S_{kejt}} \pi_{kjt}+\pi_{ijt} & l\leq N_1 \\
		1-\sum_k\theta_{ke}\overline{S_{kejt}}    & N_1<l\leq N_1+N_2	 \\
		\Delta  \left[\ln \frac{q_{ejt+1}}{q_{e'jt+1}}\right]-\beta_j\left[\sum_k\left(\theta_{ke} \overline{S_{kjet}}-\theta_{ke'} \overline{S_{kje' t}}\right)\pi_{kjt}\right]-\gamma_{e,e't} & N_1+N_2<l\leq N 
	\end{array}\right.
\eeqns

Note that the parameter vector $\mu$ has four types of parameters: $\theta_{ke}$, $\pi_{kjt}$, and $\beta_j$. Let us go case by case:
\beqns
	\frac{\partial\varepsilon_l(\mu)}{\partial \beta_j}&=&\left\{\begin{array}{lc}
		-\beta_j\left[\sum_k\left(\theta_{ke} \overline{S_{kjet}}-\theta_{ke'} \overline{S_{kje' t}}\right)\pi_{kjt}\right] & N_1+N_2<l\leq N, j=J(l)\\
		0& \text{otherwise}
	\end{array}\right.\\
\frac{\partial\varepsilon_l(\mu)}{\partial \gamma_{ee't}}&=&\left\{\begin{array}{lc}
	-1 & N_1+N_2<l\leq N, (e,e',t)=EET(l)\\
	0&  \text{otherwise}
\end{array}\right.\\
\frac{\partial\varepsilon_l(\mu)}{\partial \pi_{ijt}}&=&\left\{\begin{array}{lc}
- \theta_{ie}\overline{S_{iejt}} & l\leq N_1, i\neq I(l), j=J(l), t=T(l) \\
- \theta_{ie}\overline{S_{iejt}}+1 & l\leq N_1, i= I(l),j=J(l), t=T(l) \\
-\beta_j\left(\theta_{ie} \overline{S_{ijet}}-\theta_{ie'} \overline{S_{ije' t}}\right) & N_1+N_2<l\leq N \\
0 & \text{otherwise}
\end{array}\right.\\
\frac{\partial \varepsilon_l(\mu)}{\partial \theta_{ie}}&=&\left\{\begin{array}{lc}
- \overline{S_{kejt}} \pi_{ijt} & l\leq N_1, e=E(l) \\
	-\overline{S_{iejt}}    & N_1<l\leq N_1+N_2,	e=E(l) \\
-\beta_j\theta_{ie}\overline{S_{ije t}}\pi_{ijt} & N_1+N_2<l\leq N, (e,\cdot,\cdot)=EET(l)\\
	\beta_j\theta_{ie} \overline{S_{ije t}}\pi_{ijt} & N_1+N_2<l\leq N, (\cdot,e,\cdot)=EET(l)\\
	0 & \text{otherwise}
\end{array}\right.
\eeqns
Let $\Xi$ be the $N\times R$ matrix with general term $\Xi_{ir}=\frac{\partial \varepsilon(w_i,c)}{\partial c_{r}}$. Then our estimate of the gradient is $P\times R$ matrix:
\beqns
	\hat{D}=\frac{1}{N}Z'\Xi
\eeqns

\end{document}

