\documentclass[a4paper, 12pt]{article}

% Margins
\topmargin=-0.45in
\evensidemargin=0in
\oddsidemargin=0in
\textwidth=6.5in
\textheight=9.0in
\headsep=0.25in 

%package usage
\PassOptionsToPackage{svgnames}{xcolor}
\usepackage[colorlinks=true, citecolor=blue, linkcolor=purple]{hyperref}
\usepackage[english]{babel}
\usepackage[latin1]{inputenc}
\usepackage{enumitem}
\usepackage{indentfirst}
\usepackage{colortbl}
\usepackage{longtable}
\usepackage{animate}
\usepackage{threeparttablex}
\usepackage{etoolbox}
\usepackage{rotating}
\usepackage{multirow}
\usepackage{pdflscape}
\usepackage{tablefootnote}
\usepackage[table,xcdraw]{xcolor}
\usepackage{amsmath}
\usepackage{flafter} 
\usepackage{dcolumn} 
\usepackage{natbib}
\usepackage{rotating}	
\usepackage{adjustbox}
\usepackage{amsthm}
\usepackage{graphicx}
\usepackage{amssymb}
\usepackage{tcolorbox}
\usepackage{lipsum}
\usepackage{tikz}
\usepackage{tabularx}
\tcbuselibrary{skins,breakable}
\usetikzlibrary{shadings,shadows}
\usepackage{threeparttable}
\usepackage{subfig}
\usepackage{setspace}
\usepackage{booktabs}
\usepackage{placeins}
\usepackage{enumitem}
\usepackage{natbib}
\usepackage{filecontents}
\usepackage[encoding,filenameencoding=utf8,extendedchars,space]{grffile}
\newcommand{\addfig}[2]{\begin{center}
			\includegraphics[width=#1\textwidth]{#2}
	\end{center}
}

%\usepackage[capposition=top]{floatrow}
%\usepackage[colorinlistoftodos]{todonotes}
\newcommand{\expect}[2]{\mathbb{E}_{#2}\left(#1\right)}

\newtheorem{theorem}{Theorem}[section]
\newtheorem{corollary}{Corollary}[theorem]
\newtheorem{proposition}[theorem]{Proposition}

\newcommand{\alert}[1]{{\textbf{\color{red}#1}}}
%general commands

\newcommand{\bcite}{\begin{quote}}
\newcommand{\ecite}{\end{quote}}
\newcommand{\beqns}{\begin{eqnarray*}}
\newcommand{\eeqns}{\end{eqnarray*}}
\newcommand{\beqn}{\begin{eqnarray}}
\newcommand{\eeqn}{\end{eqnarray}}
\newcommand{\benu}{\begin{enumerate}}
\newcommand{\eenu}{\end{enumerate}}
\newcommand{\bitem}{\begin{itemize}}
\newcommand{\eitem}{\end{itemize}}
\newcommand{\smallGap}{\vspace{.25cm}}

\newenvironment{block}[1]{%
	\tcolorbox[beamer,%
	noparskip,breakable,
	colback=LightGreen,colframe=DarkGreen,%
	colbacklower=LimeGreen!75!LightGreen,%
	title=#1]}%
{\endtcolorbox}


\newcommand{\sym}[1]{\rlap{#1}}% Thanks David Carlisle

\usepackage{siunitx}
\sisetup{
	detect-mode,
	group-digits		= false,
	input-symbols		= ( ) [ ] - +,
	table-align-text-post	= false,
	input-signs             = ,
}

%mathematical commands
\newcommand{\problemIndustries}{(A_t\cup \Omega_t)}
\newcommand{\red}[1]{{\color{red}#1}}
\newcommand{\normalIndustries}{\problemIndustries^C}
\newcommand{\sumNormalIndustries}{\sum_{j\in\normalIndustries}}
\newcommand{\sumProblemIndustries}{\sum_{j\in\problemIndustries}}
\newcommand{\sumWithinNormal}{\sum_{j\in\normalIndustries\cap \Gamma_t}}
\newcommand{\sumWithinProblem}{\sum_{j\in\normalIndustries\cap \Gamma_t^C}}
\newcommand{\ser}[1]{s_{er#1}}
\newcommand{\cov}{\text{cov}}
\newcommand{\explain}[2]{\underbrace{#1}_{\parbox{\widthof{\ensuremath{#1}}}{\footnotesize\raggedright #2}}}
\newcommand{\lfpthresh}[1]{\underline{\chi_{#1 lt}}}
\newcommand{\crho}{\frac{\sigma-1}{\sigma}}
\newcommand{\crhoinv}{\frac{\sigma}{\sigma-1}}

\newcommand{\lquote}[3][]{\bcite #2 \citep[#1]{#3} \ecite}

\newcommand{\laquote}[3][]{\bcite #2 \citepalias[#1]{#3} \ecite}
\usepackage{tabularx}
\defcitealias{NationalAcademyofSciences.CommitteeonOccupationalClassificationandAnalysis.1971}{National Academy of Sciences, 1971}


\title{Code documentation}
\author{C\'esar Garro-Mar\'in\thanks{Boston University, email: \href{mailto:cesarlgm@bu.edu}{cesarlgm@bu.edu}}} 
\begin{document}
\maketitle

\newcommand{\ntimes}{4 }
\section{Estimating $\theta$}
\subsection{Assuming scales}
We normalize all the skill questions to range between zero and one. I define as the simple average of the skill questions involved:
\beqns
	S_{\theta,m}=\frac{1}{||m||}\sum_{l=1}^{||m||}s_{mli}
\eeqns
where $s_{mji}$ is the individual $i$'s answer to the skill question $l$. Next, I aggregate the dataset to at the occupation-year level and I estimate $\theta_i$ using the regression:
\beqns
		d\ln f_\theta(J)-\overline{d\ln f_\theta(J)}&=&\pi_\theta+\sum_{i}\beta_{\theta,m}(S^\star_{\theta,m}(J)-\overline{S^\star_{\theta,m}(J)})+\nu_{\theta}(J)
\eeqns
The implied $\theta_i$ are given by,
\bitem 
	\item Unweighted: \href{https://www.dropbox.com/s/epmal84fzmnwiam/unweighted_thetas.txt?dl=0}{see log file}
	\item Weighted: \href{https://www.dropbox.com/s/bkq8o6zcjmgpjf8/weighted_thetas.txt?dl=0}{see log file}
\eitem 

\begin{center}
\begin{threeparttable}[!h]
\caption{Estimates of $\beta_{i}^e$}
\begin{tabular}{lcc}
\toprule
\toprule
&\multicolumn{1}{c}{\textbf{Unweighted}}&\multicolumn{1}{c}{\textbf{Weighted}} \\
\textbf{}&\multicolumn{1}{c}{(1)}&\multicolumn{1}{c}{(2)} \\
\midrule
manual1             &        0.03&        0.03\\
                    &      (0.11)&      (0.11)\\
abstract1           &       -0.69&       -0.69\\
                    &      (0.26)&      (0.26)\\
social1             &       -0.01&       -0.01\\
                    &      (0.29)&      (0.29)\\
routine1            &       -0.19&       -0.19\\
                    &      (0.18)&      (0.18)\\
manual2             &        0.58&        0.58\\
                    &      (0.28)&      (0.28)\\
abstract2           &       -2.78&       -2.78\\
                    &      (0.60)&      (0.60)\\
social2             &        2.31&        2.31\\
                    &      (0.66)&      (0.66)\\
routine2            &        0.56&        0.56\\
                    &      (0.37)&      (0.37)\\
manual3             &        0.06&        0.06\\
                    &      (0.13)&      (0.13)\\
abstract3           &        0.17&        0.17\\
                    &      (0.15)&      (0.15)\\
social3             &       -0.11&       -0.11\\
                    &      (0.18)&      (0.18)\\
routine3            &       -0.62&       -0.62\\
                    &      (0.24)&      (0.24)\\
n\_occupations       &            &            \\
N                   &       4,687&       4,687\\
r2                  &        0.07&        0.07\\
\bottomrule
\bottomrule
\end{tabular}
\begin{tablenotes}
\item \footnotesize \textit{Notes:} Standard errors clustered at the occupation level. Estimates include industry by education fixed-effects. Table generated on  8 Mar 2022 at 16:33:20.
\end{tablenotes}
\end{threeparttable}
\end{center}



\subsection{Sanity check}
\subsubsection{Simulating data}
If the model were true in the data, it generate data following the two equations below:
\beqn
\sum_{m=1}^I\theta_jS_{\theta,m}(J)&=&1 \label{eq:skill_sum}\\
d\ln f_\theta(J)-\overline{d\ln f_\theta(J)}&=&\frac{\varepsilon}{\varepsilon-1}\sum_i\left[\theta_iS^\star_{\theta,i}(J)-\theta_i\overline{S^\star_{\theta,i}(J)}\right]d\ln A_i  \label{eq:skill_reg}
\eeqn
\textbf{How do I simulate the data:}

Equation \eqref{eq:skill_reg} is simple. Choose $\varepsilon$ and $d\ln A_i$ and generate data following:
\beqns
d\ln f_\theta(J)-\overline{d\ln f_\theta(J)}&=&\frac{\varepsilon}{\varepsilon-1}\sum_i\left[\theta_iS^\star_{\theta,i}(J)-\overline{S^\star_{\theta,i}(J)}\right]d\ln A_i+\nu_\theta(J)
\eeqns
Making equation \eqref{eq:skill_sum} work seems more involved. This equation implies that:
\beqns
	S_{\theta,M}(J)=\frac{1}{\theta_M}\left(1-\sum_{m=1}^{M-1}\theta_{i}S_{\theta,m}(J)\right)+\eta_{\theta}(J)
\eeqns

I will start simple:
\bitem
	\item Assume a matrix of $\theta_i$.
	\item Why am I complicating myself with this? Simply generate the data following the above equation.
	\item See if my algorithm works in finding the solution.
\eitem 

\subsection{Estimating scales}
%Fix the notation here
Estimation uses two key equations from the model:
\beqn
	\sum_{m=1}^I\theta_jS_{\theta,m}(J)&=&1 \label{eq:skill_sum}\\
	d\ln f_\theta(J)-\overline{d\ln f_\theta(J)}&=&\frac{\varepsilon}{\varepsilon-1}\sum_i\left[\theta_iS^\star_{\theta,i}(J)-\theta_i\overline{S^\star_{\theta,i}(J)}\right]d\ln A_i
\eeqn
our current procedure to estimate the model parameter is choose scales $c_{jml}$ and scale weights $\alpha_{jm}$  to minimize the MSE from equation \eqref{eq:skill_sum}:
		\beqns
		\min_{\alpha_{jm},c_{jml}}\frac{1}{N}\left[\sum_{m=1}^I\theta_jS_{\theta,m}-1\right]^2 \text{ s.t. }  S_{\theta,m}=\sum_{j=1}^{||m||}\alpha_{jm}\sum_{l=1}^5c_{jml}1_{d_{ijm}=l}
		\eeqns
 Where $\theta_i$ comes from an OLS regression using the estimating equation:
	\beqns
		d\ln f_\theta(J)-\overline{d\ln f_\theta(J)}&=&\pi_\theta+\sum_{i}\beta_{\theta,i}(S^\star_{\theta,i}(J)-\theta_i\overline{S^\star_{\theta,i}(J)})+\nu_{\theta}(J)
	\eeqns



\section{Defining education groups}
	Our current results group education levels into three broad groups that I will often call Low, Mid, and High.
% Table generated by Excel2LaTeX from sheet 'Sheet1'
\begin{table}[htbp]
	\centering
	\caption{Add caption}
	\begin{tabular}{ll}
		\toprule
		\textbf{ Label } & \textbf{ GCSE qualification level } \\
		\midrule
		Low   & Below GCSE A \\
		Mid   & GCSE A* / trade qualification \\
		High  & Bachelor +  \\
		\bottomrule
		\bottomrule
	\end{tabular}%
	\label{tab:addlabel}%
\end{table}%

	

\section{Classifying jobs}
We say that an occupation $j$ is a core job of education group $e$ if two conditions are met:
\benu	 
	\item Education group $e$ is overrepresented in the occupation relative to the overall population. That is:
	\beqns
		s_e(j)\geq\overline{s}_e
	\eeqns
	where $s_e(j)$ denotes the employment share of the education group $e$ in job $j$, and $\overline{s}_e$ is its employment share in the population.
	\item The employment share of group $e$ in job $j$ is at least \ntimes the employment share of any other education group that is overrepresented in the occupation.
	\beqns
		s_e(j)\geq\ntimes s_{e'}(j)
	\eeqns
	for any other education group $e'$ such that $	s_{e'}(j)\geq\overline{s}_{e'}$.
\eenu

\section{Computing $\theta$s}
\subsection{Data I use}
First I restrict data to only:
\benu 
\item occupations that are core jobs in two consecutive SES-waves.
\item people with education levels matching the job-classification. For example, I restrict to observations of individuals with low-education in low-education core-jobs.
\eenu 
Using this restricted dataset, I occupational employment shares by education level:
\beqns
	s_e(j)=\frac{l_e(j)}{\sum_{j'}l_e(j')}
\eeqns
where $l$ denotes employment and the summation is over jobs that stay in the core of education group $e$ in two consecutive SES-waves.


\section{Solution procedure}

Out of equation 32 we have:
\beqns
	\frac{\partial \ln f_\theta(J)}{\partial A_i}-\frac{\partial \ln f_\theta(J')}{\partial A_i}&=&\frac{\varepsilon}{\varepsilon-1}\left[\frac{\ln y^\star_\theta(J)}{\partial \ln A_i}-\frac{\ln y^\star_\theta(J')}{\partial \ln A_i}\right]
\eeqns
Moreover, out of question 44 we have
\beqns
\frac{\partial \ln y^\star_\theta(J)}{\partial \ln A_i}&=&\theta_iS^\star_{\theta,i}(J)
\eeqns
Plugging into 32 we have:
\beqn
\label{eq:plug}
\frac{\partial \ln f_\theta(J)}{\partial A_i}-\frac{\partial \ln f_\theta(J')}{\partial A_i}&=&\frac{\varepsilon}{\varepsilon-1}\left[\theta_iS^\star_{\theta,i}(J)-\theta_iS^\star_{\theta,i}(J')\right]
\eeqn
Thus:
\beqns
d\ln f_\theta(J)-d\ln f_\theta(J')=\frac{\varepsilon}{\varepsilon-1}\sum_i\left[\theta_iS^\star_{\theta,i}(J)-\theta_iS^\star_{\theta,i}(J')\right]d\ln A_i+\theta_MS^\star_{\theta,M}(J)-\theta_MS^\star_{\theta,M}(J')
\eeqns
Summing over jobs and dividing by the number of jobs we have:
\beqns
d\ln f_\theta(J)-\overline{d\ln f_\theta(J)}=\frac{\varepsilon}{\varepsilon-1}\sum_i\left[\theta_iS^\star_{\theta,i}(J)-\theta_i\overline{S^\star_{\theta,i}(J)}\right]d\ln A_i
\eeqns
This equation calls for the following regression specification:
\beqns
	d\ln f_\theta(J)-\overline{d\ln f_\theta(J)}&=&\alpha_\theta+\sum_{i}\beta_{\theta,i}(S^\star_{\theta,i}(J)-\theta_i\overline{S^\star_{\theta,i}(J)})+\nu_{\theta}(J)
\eeqns
Then:
\bitem
	\item Under the assumption that $\theta_i=1$, $\frac{\varepsilon}{\varepsilon-1}d\ln A_i$ is identified out of the low education group.
	\item Rest of education groups identify $\theta_i$. 
\eitem 





\subsection{Procedure}
\benu
	\item Guess $S_{\theta,i}(J)$.
	\item Estimate $\theta_i$ out of core jobs. %\red{What is the reference job? The average for that education group?}
	\item Given $\theta_i$ estimate $S_{\theta,i}(J)$.
	\item Return to 1.
\eenu

\subsection{What functions do I need to write}
\subsubsection{Estimation of $\theta_i$}
Let $y_\theta$ be the $J\times 1$ vector containing the vector of $d\ln f_\theta(J)-d\ln f_\theta(J')$. Let $S_\theta$ the $J\times I$ matrix of skill indexes $S^\star_{\theta,i}(J)-S^\star_{\theta,i}(J')$. Then:
\beqns
	\beta_\theta=\frac{\epsilon}{\epsilon-1}[\theta_1d\ln A_1 \dots \theta_Id\ln A_I]'
\eeqns
I estimate $\beta_\theta$ by OLS:
\beqns
	\beta_\theta=(X_\theta'X_\theta)^{-1}X_\theta y_\theta
\eeqns
Using the appropriate block diagonal matrix I can estimate all the vectors at the same time. For this I need:
\bitem
\item The usual OLS function
\item The function to block diagonalize the matrix that I already wrote.
\eitem 
Next, I need to back out the $\theta_i$. For this I need to do:
\beqns
	\beta_1=\frac{\epsilon}{\epsilon-1}[d\ln A_1 \dots d\ln A_I]'
\eeqns
Then:
\beqns
	\theta=\beta_\theta\oslash\beta_1
\eeqns
For this I need:
\bitem
	\item Function splitting the vector by education level.
	\item Function estimating $\beta_{\theta}$: {\tt estimate\_beta\_theta}
	\item Function estimating the $\theta$ {\tt estimate\_theta}.
	\item Function estimating averages of skill indexes by education level: {\tt average\_skill\_use}.
\eitem 
\subsubsection{How do I estimate the scales then?}
There are a set of $O$ skill questions in the SES survey that we have partitioned into $M$ mutually exclusive groups that we index by $m$. Within each partition, we index the skill questions by $j$. Let $d_{ijm}$ be individual's $i$ answer for the skill question $jm$. $d_{ijm}\in\{1,2,3,4,5\}$. The problem is then:
\beqns
	\min_{\alpha_{jm},c_{jml}}\frac{1}{N}\left[\sum_{m=1}^I\theta_jS_{\theta,m}\right]^2 \text{ s.t. }  S_{\theta,m}=v
\eeqns
\bitem
	\item I think this is mostly done. I just have to modify the loss function for this.
\eitem 
\bibliographystyle{apalike}
\bibliography{../../../../CentralLibrary/Papers/library}{}

\end{document}

