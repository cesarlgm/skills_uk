\documentclass[a4paper, 12pt]{article}

% Margins
\topmargin=-0.45in
\evensidemargin=0in
\oddsidemargin=0in
\textwidth=6.5in
\textheight=9.0in
\headsep=0.25in 

%package usage
\PassOptionsToPackage{svgnames}{xcolor}
\usepackage[colorlinks=true, citecolor=blue, linkcolor=purple]{hyperref}
\usepackage[english]{babel}
\usepackage[latin1]{inputenc}
\usepackage{enumitem}
\usepackage{indentfirst}
\usepackage{colortbl}
\usepackage{longtable}
\usepackage{animate}
\usepackage{threeparttablex}
\usepackage{etoolbox}
\usepackage{rotating}
\usepackage{multirow}
\usepackage{pdflscape}
\usepackage{tablefootnote}
\usepackage[table,xcdraw]{xcolor}
\usepackage{amsmath}
\usepackage{flafter} 
\usepackage{dcolumn} 
\usepackage{natbib}
\usepackage{rotating}	
\usepackage{amsthm}
\usepackage{graphicx}
\usepackage{amssymb}
\usepackage{tcolorbox}
\usepackage{lipsum}
\usepackage{tikz}
\usepackage{tabularx}
\tcbuselibrary{skins,breakable}
\usetikzlibrary{shadings,shadows}
\usepackage{threeparttable}
\usepackage{subfig}
\usepackage{setspace}
\usepackage{booktabs}
\usepackage{placeins}
\usepackage{enumitem}
\usepackage{natbib}
\usepackage{filecontents}
\usepackage[encoding,filenameencoding=utf8,extendedchars,space]{grffile}
\newcommand{\addfig}[2]{\begin{center}
			\includegraphics[width=#1\textwidth]{#2}
	\end{center}
}

%\usepackage[capposition=top]{floatrow}
%\usepackage[colorinlistoftodos]{todonotes}
\newcommand{\expect}[2]{\mathbb{E}_{#2}\left(#1\right)}

\newtheorem{theorem}{Theorem}[section]
\newtheorem{corollary}{Corollary}[theorem]
\newtheorem{proposition}[theorem]{Proposition}

\newcommand{\alert}[1]{{\textbf{\color{red}#1}}}
%general commands

\newcommand{\bcite}{\begin{quote}}
\newcommand{\ecite}{\end{quote}}
\newcommand{\beqns}{\begin{eqnarray*}}
\newcommand{\eeqns}{\end{eqnarray*}}
\newcommand{\beqn}{\begin{eqnarray}}
\newcommand{\eeqn}{\end{eqnarray}}
\newcommand{\benu}{\begin{enumerate}}
\newcommand{\eenu}{\end{enumerate}}
\newcommand{\bitem}{\begin{itemize}}
\newcommand{\eitem}{\end{itemize}}
\newcommand{\smallGap}{\vspace{.25cm}}

\newenvironment{block}[1]{%
	\tcolorbox[beamer,%
	noparskip,breakable,
	colback=LightGreen,colframe=DarkGreen,%
	colbacklower=LimeGreen!75!LightGreen,%
	title=#1]}%
{\endtcolorbox}


\newcommand{\sym}[1]{\rlap{#1}}% Thanks David Carlisle

\usepackage{siunitx}
\sisetup{
	detect-mode,
	group-digits		= false,
	input-symbols		= ( ) [ ] - +,
	table-align-text-post	= false,
	input-signs             = ,
}

%mathematical commands
\newcommand{\problemIndustries}{(A_t\cup \Omega_t)}
\newcommand{\red}[1]{{\color{red}#1}}
\newcommand{\normalIndustries}{\problemIndustries^C}
\newcommand{\sumNormalIndustries}{\sum_{j\in\normalIndustries}}
\newcommand{\sumProblemIndustries}{\sum_{j\in\problemIndustries}}
\newcommand{\sumWithinNormal}{\sum_{j\in\normalIndustries\cap \Gamma_t}}
\newcommand{\sumWithinProblem}{\sum_{j\in\normalIndustries\cap \Gamma_t^C}}
\newcommand{\ser}[1]{s_{er#1}}
\newcommand{\cov}{\text{cov}}
\newcommand{\explain}[2]{\underbrace{#1}_{\parbox{\widthof{\ensuremath{#1}}}{\footnotesize\raggedright #2}}}
\newcommand{\lfpthresh}[1]{\underline{\chi_{#1 lt}}}
\newcommand{\crho}{\frac{\sigma-1}{\sigma}}
\newcommand{\crhoinv}{\frac{\sigma}{\sigma-1}}

\newcommand{\lquote}[3][]{\bcite #2 \citep[#1]{#3} \ecite}

\newcommand{\laquote}[3][]{\bcite #2 \citepalias[#1]{#3} \ecite}
\usepackage{tabularx}
\defcitealias{NationalAcademyofSciences.CommitteeonOccupationalClassificationandAnalysis.1971}{National Academy of Sciences, 1971}


\title{Results with simple average indexes}
\author{C\'esar Garro-Mar\'in\thanks{Boston University, email: \href{mailto:cesarlgm@bu.edu}{cesarlgm@bu.edu}}} 
\begin{document}
	\maketitle
	
	\newcommand{\ntimes}{4 }
	
	Here I show estimated skill acquisition costs when I compute them using simple average indexes. These $\theta_i$ come from the regression:
	\beqn
	d\ln f^e(J)&=&\sum_i\beta_i^eS_i^e(J)\\
	\eeqn
	where:
	\beqns
		\beta_{i}^e=\frac{\varepsilon}{1-\varepsilon}\theta_i^e(d\ln A_i - K^e)
	\eeqns
	I define the indexes as follows:
	\beqns
		S_i(J)&=&\frac{\tilde{S}_i(J)}{\sum_k^K\tilde{S}_k(J) } \\
	\eeqns
	where $S_k(J)$ is the simple average of the scores I assigned to each SES question:
	\beqns
		S_{i}(J)=\frac{1}{||i||}\sum_{j=1}^{||i||}\sum_{l=1}^5c_{ijl}1_{d_{ij}=l}\\
	\eeqns
	remember that the SES questions have possible answer going from 1 to 5. I normalized these answers $c_{ijl}$  to be between zero and one.
	\beqns
		c_{ijl}=\frac{l-1}{4}
	\eeqns
	\section{Results}
	Weighted result weights observation by the occupation-years cells size. The implied thetas are in these files:
	\bitem
		\item Unweighted $\theta$s: \href{URL}{click here}
		\item Weighted $\theta$s: \href{URL}{click here}
	\eitem
	\FloatBarrier
	\begin{center}
\begin{threeparttable}[!h]
\caption{Estimates of $\beta_{i}^e$}
\begin{tabular}{lcc}
\toprule
\toprule
&\multicolumn{1}{c}{\textbf{Unweighted}}&\multicolumn{1}{c}{\textbf{Weighted}} \\
\textbf{}&\multicolumn{1}{c}{(1)}&\multicolumn{1}{c}{(2)} \\
\midrule
manual1             &        0.03&        0.03\\
                    &      (0.11)&      (0.11)\\
abstract1           &       -0.69&       -0.69\\
                    &      (0.26)&      (0.26)\\
social1             &       -0.01&       -0.01\\
                    &      (0.29)&      (0.29)\\
routine1            &       -0.19&       -0.19\\
                    &      (0.18)&      (0.18)\\
manual2             &        0.58&        0.58\\
                    &      (0.28)&      (0.28)\\
abstract2           &       -2.78&       -2.78\\
                    &      (0.60)&      (0.60)\\
social2             &        2.31&        2.31\\
                    &      (0.66)&      (0.66)\\
routine2            &        0.56&        0.56\\
                    &      (0.37)&      (0.37)\\
manual3             &        0.06&        0.06\\
                    &      (0.13)&      (0.13)\\
abstract3           &        0.17&        0.17\\
                    &      (0.15)&      (0.15)\\
social3             &       -0.11&       -0.11\\
                    &      (0.18)&      (0.18)\\
routine3            &       -0.62&       -0.62\\
                    &      (0.24)&      (0.24)\\
n\_occupations       &            &            \\
N                   &       4,687&       4,687\\
r2                  &        0.07&        0.07\\
\bottomrule
\bottomrule
\end{tabular}
\begin{tablenotes}
\item \footnotesize \textit{Notes:} Standard errors clustered at the occupation level. Estimates include industry by education fixed-effects. Table generated on  8 Mar 2022 at 16:33:20.
\end{tablenotes}
\end{threeparttable}
\end{center}

	\FloatBarrier
	\end{document}
	
