\begin{center}
\begin{threeparttable}[!h]
\caption{Polarizing occupations: change in occupational employment shares by education group, 2001-2017}
\label{tab:pol_occ}
\begin{tabular}{lccc}
\toprule
\toprule
&\multicolumn{1}{c}{\textbf{Low}}&\multicolumn{1}{c}{\textbf{Mid}}&\multicolumn{1}{c}{\textbf{High}} \\
\textbf{Occupation}&\multicolumn{1}{c}{(1)}&\multicolumn{1}{c}{(2)}&\multicolumn{1}{c}{(3)} \\
\midrule
2121 civil, mechanical, electrical and electronics engineers&        0.02&       -0.05&        0.03\\
2129 engineering professionals n.e.c.&        0.05&       -0.12&        0.07\\
2434 chartrd surveyors (not qntity surv)&        0.05&       -0.04&       -0.01\\
5213 Metal forming, welding and related trades&        0.07&       -0.10&        0.03\\
5312 bricklayers, masons, roofers&        0.08&       -0.14&        0.06\\
5421 printing trades&        0.08&       -0.18&        0.10\\
5431 butchers, meat cutters&        0.07&       -0.09&        0.02\\
6221 Hairdressers And Related Occupations&        0.06&       -0.15&        0.09\\
8215 Transport operatives nec&        0.06&       -0.11&        0.05\\
\bottomrule
\bottomrule
\end{tabular}
\begin{tablenotes}
\item \footnotesize \textit{Notes:} the table shows the 9 occupations (i) where the low education share increased, and (ii) survived the Benjamini-Hochberg step up procedure for the test of no change in the low education employment share at 20\% significance level. Table generated on  1 Apr 2022 at 10:44:30.
\end{tablenotes}
\end{threeparttable}
\end{center}
