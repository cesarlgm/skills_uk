\begin{center}
\begin{threeparttable}[!h]
\caption{Polarizing occupations: change in occupational employment shares by education group, 2001-2017}
\label{tab:pol_occ}
\begin{tabular}{lccc}
\toprule
\toprule
&\multicolumn{1}{c}{\textbf{Low}}&\multicolumn{1}{c}{\textbf{Mid}}&\multicolumn{1}{c}{\textbf{High}} \\
\textbf{Occupation}&\multicolumn{1}{c}{(1)}&\multicolumn{1}{c}{(2)}&\multicolumn{1}{c}{(3)} \\
\midrule
5213 Metal forming, welding and related trades&        0.07&       -0.10&        0.03\\
5241 Electrical trades&        0.01&       -0.15&        0.14\\
5312 bricklayers, masons, roofers&        0.08&       -0.14&        0.06\\
5321 building trades&        0.06&       -0.09&        0.03\\
8135 tyre, exhaust and windscrn fitters&        0.06&       -0.05&       -0.01\\
8214 taxi, cab drivers and chauffeurs&        0.01&       -0.06&        0.04\\
8215 Transport operatives nec&        0.06&       -0.11&        0.05\\
\bottomrule
\bottomrule
\end{tabular}
\begin{tablenotes}
\item \footnotesize \textit{Notes:} the table shows the 9 occupations (i) where the low education share increased, and (ii) survived the Benjamini-Hochberg step up procedure for the test of no change in the low education employment share at 20\% significance level. Table generated on  3 May 2022 at 15:00:08.
\end{tablenotes}
\end{threeparttable}
\end{center}
