\begin{center}
\begin{threeparttable}[!h]
\caption{Effect of relaxing number of switches constraint}
\begin{tabular}{lccccccc}
\toprule
\toprule
& \multicolumn{7}{c}{\textbf{Definition}} \\
\cline{2-8}
&\multicolumn{1}{c}{\textbf{3-3-1}}&\multicolumn{1}{c}{\textbf{4-4-3}}&\multicolumn{1}{c}{\textbf{3-5-3}}&\multicolumn{1}{c}{\textbf{5-3-3}}&\multicolumn{1}{c}{\textbf{5-5-7}}&\multicolumn{1}{c}{\textbf{4-6-7}}&\multicolumn{1}{c}{\textbf{6-4-7}} \\
\textbf{Transition type}&\multicolumn{1}{c}{(1)}&\multicolumn{1}{c}{(2)}&\multicolumn{1}{c}{(3)}&\multicolumn{1}{c}{(4)}&\multicolumn{1}{c}{(5)}&\multicolumn{1}{c}{(6)}&\multicolumn{1}{c}{(7)} \\
\midrule
\midrule
Low to Low-Mid&9&14&15&16&18&18&18 \\
Mid to Low-Mid&1&1&1&1&1&1&2 \\
Low-Mid to Mid&1&1&1&1&1&1&1 \\
Mid-High to High&2&2&3&3&3&3&3 \\
Total&13&18&20&21&23&23&24 \\
\bottomrule
\bottomrule
\end{tabular}
\begin{tablenotes}
\item \footnotesize \textit{Note:} each column shows the breakdown by transition type when my definition of a transitioning job is the union all the current and previous columns. For example in column two I define a transition job as the union of 3-3-1 and 4-4-3. Table generated on 10 Jun 2020 at 17:59:46.
\end{tablenotes}
\end{threeparttable}
\end{center}
