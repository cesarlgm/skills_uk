\begin{center}
\begin{threeparttable}[!h]
\caption{Relative skill use across education groups (simple average indexes)}
\label{tab:skillRegs}
\begin{tabular}{lccc}
\toprule
\toprule
&\multicolumn{1}{c}{\textbf{Analytical}}&\multicolumn{1}{c}{\textbf{Manual}}&\multicolumn{1}{c}{\textbf{Routine}} \\
\textbf{}&\multicolumn{1}{c}{(1)}&\multicolumn{1}{c}{(2)}&\multicolumn{1}{c}{(3)} \\
\midrule
GCSE C-A levels     &       0.034\sym{***}&      -0.018\sym{**} &       0.043\sym{***}\\
                    &     (0.005)         &     (0.006)         &     (0.011)         \\
\hspace{3mm}\textit{Effect size}\vspace{4mm}&       0.293         &      -0.090         &       0.254         \\
\midrule Bachelor+  &       0.072\sym{***}&      -0.081\sym{***}&      -0.039\sym{**} \\
                    &     (0.005)         &     (0.007)         &     (0.013)         \\
\hspace{3mm}\textit{Effect size}&       0.618         &      -0.409         &      -0.227         \\
\midrule Overall $ R^2$&        0.35         &        0.44         &        0.13         \\
Observations        &      14,592         &      14,592         &      14,592         \\
\bottomrule
\bottomrule
\end{tabular}
\begin{tablenotes}
\item \footnotesize \textit{Note:} all skill indexes range between 0 and 1. Regressions use individual-level data. Robust standard errors in parenthesis. Coefficents represent the difference relative the lower education level. I use dummy of basic to moderate PC use complexity as measure of routineness. I pool data from all years. Regressions include occupation fixed-effects. Effect sizes are computed as the regression coefficient divided by the standard deviation in the occupation-level skill indexes. Table generated on 12 Jun 2020 at 17:15:28. Table generated with do file 3\_sesAnalysis/createSESSkillRegressionsPooled.do
\end{tablenotes}
\end{threeparttable}
\end{center}
