\begin{center}
\begin{threeparttable}[!h]
\caption{Intersection between core and transition type definitions}
\begin{tabular}{lccccc}
\toprule
\toprule
& \multicolumn{5}{c}{\textbf{Years as core}} \\
\cline{2-6}
\textbf{Core type}&\multicolumn{1}{c}{\textbf{17}}&\multicolumn{1}{c}{\textbf{16}}&\multicolumn{1}{c}{\textbf{15}}&\multicolumn{1}{c}{\textbf{14}}&\multicolumn{1}{c}{\textbf{13}} \\
\midrule
\midrule
Below GCSE C&0&0&0&0&2 \\
GCSE C to A lev.&0&0&1&1&1 \\
Bachelor +&0&0&1&1&3 \\
Below GCSE C - GCSE C to A lev.&0&0&0&3&6 \\
Below GCSE C - Bach+&0&0&0&0&0 \\
GCSE C to A lev. - Bach+&0&0&0&1&1 \\
Total&0&0&2&6&13 \\
\bottomrule
\bottomrule
\end{tabular}
\begin{tablenotes}
\item \footnotesize \textit{Note:} transitioning jobs are defined as the union of 3-3-3, 2-4-3 and 4-2-3 definitions. Table generated on 12 Jun 2020 at 12:36:13.
\end{tablenotes}
\end{threeparttable}
\end{center}
