\begin{center}
\begin{threeparttable}[!h]
\caption{Effect of relaxing number of switches constraint}
\begin{tabular}{lcccccc}
\toprule
\toprule
& \multicolumn{6}{c}{\textbf{Definition}} \\
\cline{2-7}
&\multicolumn{1}{c}{\textbf{3-3-3}}&\multicolumn{1}{c}{\textbf{2-4-3}}&\multicolumn{1}{c}{\textbf{4-2-3}}&\multicolumn{1}{c}{\textbf{5-5-7}}&\multicolumn{1}{c}{\textbf{4-6-7}}&\multicolumn{1}{c}{\textbf{6-4-7}} \\
\textbf{Transition type}&\multicolumn{1}{c}{(1)}&\multicolumn{1}{c}{(2)}&\multicolumn{1}{c}{(3)}&\multicolumn{1}{c}{(4)}&\multicolumn{1}{c}{(5)}&\multicolumn{1}{c}{(6)} \\
\midrule
\midrule
Low to Low-Mid&17&18&20&22&22&22 \\
Mid to Low-Mid&1&1&1&1&1&2 \\
Mid to Mid-High&0&0&1&1&1&1 \\
Low-Mid to Mid&1&1&1&1&1&1 \\
Mid-High to High&3&4&4&4&4&4 \\
Total&22&24&27&29&29&30 \\
\bottomrule
\bottomrule
\end{tabular}
\begin{tablenotes}
\item \footnotesize \textit{Note:} each column shows the breakdown by transition type when my definition of a transitioning job is the union all the current and previous columns. Table generated on 10 Jun 2020 at 17:59:46.
\end{tablenotes}
\end{threeparttable}
\end{center}
